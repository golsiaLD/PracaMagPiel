% Q14
% pytanie: 14 - 14.Czy czuje Pan/i, dylemat wewnętrzny, wynikający z konieczności ciągłego podnoszenia kwalifikacji i wiążącymi się z nim ograniczeniami w życiu osobistym?
%Na pytanie: \textit{,,Czy czuje Pan/i dylemat wewnętrzny, wynikający z konieczności ciągłego podnoszenia kwalifikacji i wiążącymi się z nim ograniczeniami w życiu osobistym?''}, odpowiedzi \textit{zdecydowanie tak} udzieliło 11,3\% ankietowanych (13 osób), \textit{raczej tak} 35,7\% (41 osób) , \textit{nie mam zdania} 7\% (8 osób), \textit{raczej nie} 33\% (38 osób) i \textit{zdecydowanie nie} 13\% (15 osób), tab. \ref{tab:Q14}.


 Konieczność ciągłego podnoszenia kwalifikacji stanowi zdecydowany dylemat dla 11,3\% (13) respondentów. Twierdzenie to jest bliskie grupie badanych, która zaznaczyła wariant \textit{raczej tak} 35,7\% (41). W opozycji, do tej tezy znajduje się odpowiedź \textit{zdecydowanie nie} 13\% (15), oraz \textit{raczej nie} 33\% (38), którą wybrali ankietowani. Badanych, którzy nie zdecydowali się na proponowane twierdzenie było 7\% (8). Rozkład odpowiedzi widoczny jest w tab.\ref{tab:Q14}

\begin{table}[H]
\caption{Dylemat badanych, wynikający z konieczności ciągłego kształcenia zawodowego}
\centering
\begin{tabular}{ | c | c | c |}
\hline
zmienna & $n$ & \% \\
\hline
zdecydowanie tak  &  13  & 11.3 \\
\hline
raczej tak  &  41  & 35.7 \\
\hline
nie mam zdania  &  8  & 7 \\
\hline
raczej nie  &  38  & 33 \\
\hline
zdecydowanie nie  &  15  & 13 \\
\hline
\end{tabular}
\label{tab:Q14}
\end{table}
