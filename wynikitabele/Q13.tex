% Q13
% pytanie: 13 - 13. Czy, w celu radzenia sobie ze stresem związanym z pracą zawodową, sięga Pan/i po używki?
%Na pytanie: \textit{,,Czy, w celu radzenia sobie ze stresem związanym z pracą zawodową, sięga Pan/i po używki?''}, 6,1\% (7 osób) odpowiedfziało \textit{zdecydowanie tak}, 31,3\% (36 osób) \textit{raczej tak}, 28,7\% (33 osoby) \textit{nie mam zdania}, 30,4\% (35 osób) \textit{raczej nie} i 3,5\% (4 osoby) \textit{zdecydowanie nie}, tab \ref{tab:Q12}.
Niwelowanie stresu, poprzez sięganie po używki wykazało 1,7\% (2) uczestników badania.  12,2\% (14)  z nich przyznało, że raczej siega po używki w tym celu. 6,1\% (7) nie miało zdania, na ten temat. Natomiast  38,3\% (44) badanych zaznaczyło odpowiedź \textit{raczej nie}, a  41,7\% (48) \textit{zdecydowanie nie}. Przedstawia to  tab \ref{tab:Q12}.

\begin{table}[H]
\caption{Stosowanie używek przez respondentów, jako metoda radzenia ze stresem}
\centering
\begin{tabular}{ | c | c | c |}
\hline
zmienna & $n$ & \% \\
\hline
zdecydowanie tak  &  2  & 1.7 \\
\hline
raczej tak  &  14  & 12.2 \\
\hline
nie mam zdania  &  7  & 6.1\\
\hline
raczej nie  &  44  & 38.3 \\
\hline
zdecydowanie nie  &  48  & 41.7 \\
\hline
\end{tabular}
\label{tab:Q13}
\end{table}
