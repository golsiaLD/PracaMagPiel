% Q3
% pytanie: 3 - 3. Ile, w przybliżeniu, wynosi miesięczny wymiar godzin, Pana/i pracy?
Tabela \ref{tab:Q3} ilustruje miesięczny wymiar czasu pracy uczestników badania. 13\% (15) zaznaczyło, że pracuje nie więcej niż 149 godzin miesięcznie. 43,5\%  (50) podało, że pracuje od 150 do 169 godzin. 24,3\% (28) wybrało wariant, od 170 do 199 godzin. Pracę w przedziale od 200 do 239 godzin oznaczyło 12,2\% (14), a 4,3\% (5) ankietowanych wykonuje działalność zawodową w przedziale od 240 do 259 godzin. Jedynie 2,6\% (3) wykazało, że pracuje powyżej 260 godzin miesięcznie.


 
\begin{table}[H]
\caption{Miesięczny wymiar czasu pracy}
\centering
\begin{tabular}{ | c | c | c |}
\hline
zmienna & $n$ & \% \\
\hline
do 149 godzin  &  15  & 13 \\
\hline
od 150 do 169 godzin  &  50  & 43.5 \\
\hline
od 170 do 199 godzin  &  28  & 24.3 \\
\hline
od 200 do 239 godzin  &  14  & 12.2 \\
\hline
od 240 do 259 godzin  &  5  & 4.3 \\
\hline
powyżej 260 godzin  &  3  & 2.6 \\
\hline
\end{tabular}
\label{tab:Q3}
\end{table}

