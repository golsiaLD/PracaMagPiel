% Q3
% pytanie: 3 - 3. Ile, w przybliżeniu, wynosi miesięczny wymiar godzin, Pana/i pracy?
Tabela \ref{tab:Q3} ilustruje odpowiedzi ankietowanych na pytanie: \textit{,,Ile, w przybliżeniu, wynosi miesięczny wymiar godzin Pana/i pracy?''}. 13\% (15 osób) zaznaczyło, że pracuje nie więcej niż 149 godzin miesięcznie. 43,5\%  (50 osób) podało, że pracuje od 150 do 169 godzin. 24,3\% (28 osób) zaznaczyło, że pracuje od 170 do 199 godzin. Pracę w przedziale od 200 do 239 godzin zaznaczyło 12,2\% (14 osób) a 4,3\% (5 osób) w przedziale od 240 do 259 godzin. 2,6\% (3 osoby) zaznaczyły, że pracują powyżej 260 godzin miesięcznie, tab. \ref{tab:Q3}.

 
\begin{table}[H]
\caption{Miesięczny ilość godzin pracy}
\centering
\begin{tabular}{ | c | c | c |}
\hline
ozmienna & $n$ & \% \\
\hline
do 149 godzin  &  15  & 13.0\% \\
\hline
od 150 do 169 godzin  &  50  & 43.5\% \\
\hline
od 170 do 199 godzin  &  28  & 24.3\% \\
\hline
od 200 do 239 godzin  &  14  & 12.2\% \\
\hline
od 240 do 259 godzin  &  5  & 4.3\% \\
\hline
powyżej 260 godzin  &  3  & 2.6\% \\
\hline
\end{tabular}
\label{tab:Q3}
\end{table}

