% Q12
% pytanie: 12 - 12. Czy potrafi Pan/i, niwelować w.w objawy, za pomocą technik psychologicznych?

%Na pytanie: \textit{,,Czy potrafi Pan/i niwelować objawy traumatyzacji za pomocą technik psychologicznych?''} 6,1\% (7 osób) odpowiedfziało \textit{zdecydowanie tak}, 31,3\% (36 osób) \textit{raczej tak}, 28,7\% (33 osoby) \textit{nie mam zdania}, 30,4\% (35 osób) \textit{raczej nie} i 3,5\% (4 osoby) \textit{zdecydowanie nie}, tab \ref{tab:Q12}.
Umiejętnością wykorzystania technik psychologicznych, w niwelowaniu objawów traumatyzacji wtórnej wykazało się 6,1\% (7) badanych. Odpowiedzieli oni \textit{zdecydowanie tak}. Przekonanych do tej opcji było 31,3\% (36) respondentów, którzy udzielili odpowiedzi \textit{raczej tak}. Nie miało zdania 28,7\% (33) uczestników badania, a  30,4\% (35) zaznaczyło \textit{raczej nie} i 3,5\% (4) \textit{zdecydowanie nie}. Dane te znajdują się w  tab. \ref{tab:Q12}.

\begin{table}[H]
\caption{Umiejętność wykorzystania technik psychologicznych}
\centering
\begin{tabular}{ | c | c | c |}
\hline
zmienna & $n$ & \% \\
\hline
zdecydowanie tak  &  7  & 6.1 \\
\hline
raczej tak  &  36  & 31.3\\
\hline
nie mam zdania  &  33  & 28.7 \\
\hline
raczej nie  &  35  & 30.4 \\
\hline
zdecydowanie nie  &  4  & 3.5\\
\hline
\end{tabular}
\label{tab:Q12}
\end{table}
