% Q22
% pytanie: 22 - 22. Czy odczuwa Pan/i potrzebę, korzystania ze wsparcia psychologicznego dla pracowników medycznych?
%Na pytanie: \textit{,,Czy odczuwa Pan/i potrzebę korzystania ze wsparcia psychologicznego dla pracowników medycznych?''}, odpowiedzi \textit{zdecydowanie tak} udzieliło 27\% ankietowanych (31 osób), \textit{raczej tak} 20\% (23 osoby) , \textit{nie mam zdania} 13\% (15 osób), \textit{raczej nie} 30,4\% (35 osób) i \textit{zdecydowanie nie} 9,6\% (11 osób), tab. \ref{tab:Q22}.


 Potrzebę wsparcia psychologicznego jednoznacznie wyraża 27\% ankietowanych (31 osób), także sondowani, którzy odpowiedzieli \textit{raczej tak} 20\% (23) optują za tą alternatywą. \textit{Nie mam zdania}, to wybór 13\% (15) badanych. \textit{Raczej nie} 30,4\% (35) i \textit{zdecydowanie nie} 9,6\% (11), zaznaczyli respondenci o odmiennym stanowisku. Prezentuje to tab. \ref{tab:Q22}.
\begin{table}[H]
\caption{Potrzeba korzystania ze wsparcia psychologicznego}
\centering
\begin{tabular}{ | c | c | c |}
\hline
zmienna & $n$ & \% \\
\hline
zdecydowanie tak  &  31  & 27 \\
\hline
raczej tak  &  23  & 20 \\
\hline
nie mam zdania  &  15  & 13 \\
\hline
raczej nie  &  35  & 30.4 \\
\hline
zdecydowanie nie  &  11  & 9.6 \\
\hline
\end{tabular}
\label{tab:Q22}
\end{table}
