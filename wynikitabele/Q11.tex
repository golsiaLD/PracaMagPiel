% Q11
% pytanie: 11 - 11.  Czy doświadcza Pan/i, zjawiska określanego jako traumatyzacja wtórna (odczuwanie, objawów psychicznych - smutek, żal, poczucie bezradności, stres) oraz objawów somatycznych, po wystąpieniu trudnej, dramatycznej sytuacji w pracy, powyżej 30 dni?
Tabela \ref{tab:Q11} przedstawia odpowiedzi na pytanie: \textit{,,Czy doświadcza Pan/i zjawiska określanego jako traumatyzacja wtórna (odczuwanie, objawów psychicznych - smutek, żal, poczucie bezradności, stres) oraz objawów somatycznych, po wystąpieniu trudnej, dramatycznej sytuacji w pracy, powyżej 30 dni?''}.
Twierdząco odpowiedziało 38,3\% ankietowanych, w tym: 6,1\% (7 osób) \textit{zdecydowanie tak} i 32,2\% (37 osób) \textit{raczej tak}. Przecząco odpowiedziało 51,3\% ankietowanych: 40\% (46 osób) \textit{raczej nie} i 11,3\% (13 osób) \textit{zdecydowanie nie}. Odpowiedź: \textit{nie mam zdania} wybrało 10,4\% (12 osób). 


\begin{table}[H]
\caption{Traumatyzacja wtórna}
\centering
\begin{tabular}{ | c | c | c |}
\hline
zmienna & $n$ & \% \\
\hline
zdecydowanie tak  &  7  & 6.1\% \\
\hline
raczej tak  &  37  & 32.2\% \\
\hline
nie mam zdania  &  12  & 10.4\% \\
\hline
raczej nie  &  46  & 40.0\% \\
\hline
zdecydowanie nie  &  13  & 11.3\% \\
\hline
\end{tabular}
\label{tab:Q11}
\end{table}