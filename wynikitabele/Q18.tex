% Q18
% pytanie: 18 - 18. Czy ma Pan/i poczucie bezpieczeństwa w pracy w zagrożeniu koronawirusem?

%Na pytanie: \textit{,,Czy ma Pan/i poczucie bezpieczeństwa w pracy w zagrożeniu koronawirusem?''}, odpowiedzi \textit{zdecydowanie tak} udzieliło 11,3\% ankietowanych (13 osób), \textit{raczej tak} 34,8\% (40 osób) , \textit{nie mam zdania} 7,8\% (9 osób), \textit{raczej nie} 26,1\% (30 osób) i \textit{zdecydowanie nie} 20\% (23 osoby), tab. \ref{tab:Q18}.

Poczucie bezpieczeństwa w pracy w zagrożeniu koronarowirusem jest obecne u 11,3\% (13) badanej grupy zawodowej. Przychyla się do tego wariantu jest 34,8\% (40) ankietowanych.  Zdecydowanie nie czuje bezpieczeństwa w pracy w podanych warunkach 20\% (23) i 26,1\% (30),badanych którzy zaznaczyli odpowiedź \textit{raczej nie}. Pozostali uczestnicy sondażu pozostali neutralni w odpowiedzi - \input wynikitabele/Q20.tex  satysfakcja z autonomii7,8\% (9 osób). Wyniki zawiera tab. \ref{tab:Q18}.

\begin{table}[H]
\caption{Bezpieczeństwo w pracy}
\centering
\begin{tabular}{ | c | c | c |}
\hline
zmienna & $n$ & \% \\
\hline
zdecydowanie tak  &  13  & 11.3\\
\hline
raczej tak  &  40  & 34.8\\
\hline
nie mam zdania  &  9  & 7.8 \\
\hline
raczej nie  &  30  & 26.1 \\
\hline
zdecydowanie nie  &  23  & 20.0 \\
\hline
\end{tabular}
\label{tab:Q18}
\end{table}
